\documentclass[11pt, english]{article}
        \usepackage{geometry}
                \geometry{
                        a4paper,total={210mm,297mm},
                        tmargin=40.8mm,
                        bmargin=40.8mm,
                        lmargin=32.6mm,
                        rmargin=32.6mm,
                }

        \usepackage{titlesec}
                \titleformat{\section}
                        {\normalfont\fontsize{18}{16}\bfseries}{\thesection}{0.5em}{}
                \titleformat{\subsection}
                        {\normalfont\fontsize{14}{16}\bfseries}{\thesubsection}{1em}{}
                \titleformat{\subsubsection}
                        {\normalfont\fontsize{11}{16}\bfseries}{\thesubsubsection}{1em}{}

        \usepackage{longtable}
        \usepackage{multirow}

        \usepackage[labelfont=bf,textfont=bf,font=small,skip=8pt]{caption}

	\setlength{\parindent}{0pt}
        \renewcommand{\baselinestretch}{1.25}
        \usepackage{setspace}

	\usepackage{amsmath}
	\usepackage{amssymb}

\begin{document}

\pagenumbering{gobble}

	\title{\textsc{AG215 Business Finance\\ Coursework Summary}}
        \author{\textsc{Lewis Britton}}
        \date{\textsc{Academic Year 2018/2019}}
        \maketitle

\newpage

\pagenumbering{roman}

	\renewcommand{\contentsname}{Table of Contents}

        \tableofcontents

\newpage

\pagenumbering{arabic}

\section{Company Valuation}

	\subsection{Capital Asset Pricing Model}

	$$\mathrm{r=r_f+\beta(r_m-r_f)}$$

	\subsection{Earnings Per Share}

		\subsubsection{Ungeared Company}

	$$\mathrm{EPS_{UG}=\frac{EBIT(1-T_C)}{N_{UG}}}$$

	$$\mathrm{T_C=\textrm{Corporate Tax}}$$

		\subsubsection{Geared Company}
	
	$$\mathrm{EPS_G=\frac{(EBIT-r_B(B_G))(1-T_C)}{N_G}}$$

	\subsection{Earnings Yield}

	$$\mathrm{EY=\frac{EPS}{P_t}}$$

	\subsection{Rate on Equity}

		\subsubsection{Ungeared Company}

	$$\mathrm{r_{S(UG)}=\frac{EBIT(1-T_C)}{V_{UG}}}$$

		\subsubsection{Geared Company}
	
	$$\mathrm{r_{S(G)}=\frac{(EBIT-r_B(B_G))(1-T_C)}{S_G}}$$

		\subsubsection{Equity Company}

	$$\mathrm{r_S=\frac{D_1(1+g)}{P_0}+g}$$

	\subsection{Rate on Debt}

	$$\mathrm{R_{B(G)}=\frac{(EBIT-r_B(B_G))(1-T_C)}{V_G}}$$
	
	\subsection{Value of Company}

		\subsubsection{Geared Company}

	$$\mathrm{V_G=S_G+B_G}$$

		\subsubsection{Ungeared Company}

	$$\mathrm{V_{UG}=P_0(N_{UG})}$$

	$$\mathrm{\therefore V_G=V_{UG}+B_G(T_C)}$$

	\subsection{Rate on Weighted Average Cost of Capital}

	$$\mathrm{r_{WACC}=r_S\left(\frac{S_G}{B_G}\right)+\left(r_B\left(\frac{B_G}{V_G}\right)(1-T_C)\right)}$$

\newpage

\section{Working Capital}

	\subsection{Annual Holding Cost}

	$$\mathrm{AHC=\frac{1}{2}QC_H}$$

	$\mathrm{C_H=\textrm{Unit Cost to Hold}}$\\
	$\mathrm{Q=\textrm{Order Quantity}}$

	\subsection{Annual Order Cost}

	$$\mathrm{AOC=\frac{D}{Q}C_0}$$

	$\mathrm{D=\textrm{Demand}}$\\
	$\mathrm{C_0=\textrm{Unit Cost to Order}}$

	$$\mathrm{\therefore TAC=(\frac{1}{2}QC_H)+(\frac{D}{Q}C_0)}$$

	\subsection{Optimal Holding Quantity}

	$$\mathrm{Q^*=\sqrt{\frac{2DC_0}{C_H}}}$$

	\subsection{Optimal Holding Period}

	$$\mathrm{\textrm{Optimal Period}=\frac{Q^*}{\frac{D}{365}}}$$

	\subsection{Optimal Cash}

	$$\mathrm{C^*=\sqrt{\frac{2(ACR)(TC)}{r}}}$$

	$\mathrm{ACR=\textrm{Annual Cash Required}}$\\
	$\mathrm{TC=\textrm{Transaction Costs}}$

	\subsection{Optimal Cash Period}

	$$\mathrm{\textrm{Optimal Period}=\frac{C^*}{\frac{ACR}{365}}}$$

	\subsection{Optimal Target Cash Balance (All Daily)}

	$$\mathrm{Z^*=\sqrt{\frac{3(TC)(\sigma^2)}{4r}}+L}$$

	$\mathrm{U^*=\textrm{Optimal Upper Cash Balance}=3Z^*-2L}$\\
	$\mathrm{U=\textrm{Upper Cash Limit}}$\\
	$\mathrm{L=\textrm{Lower Cash Limit}}$\\
	$\mathrm{}L=\textrm{Lower Cash Limit}$\\
	$\mathrm{\sigma^*=\textrm{Variance of CFs}}$\\
	$\mathrm{r=\sqrt[365]{EAR+1}-1}$\\
	$\mathrm{\therefore \textrm{Average Cash}=\frac{4Z-L}{3}}$

\newpage

\section{Capital Budgeting \& Leasing}

	\subsection{Basic Capital Budget}

	\begin{itemize}
		\item Initial Costs
		\item Maintenance Costs
		\item Tax Savings on Maintenance Costs
		\item Scrap Value
		\item Tax Savings on Scrap Value
	\end{itemize}

		\subsubsection{Tax Saving}

	$$\mathrm{\textrm{Tax Saving}=\textrm{Tax Depreciation}*T_C }$$

		\subsubsection{Straight Line Depreciation}

	$$\mathrm{\textrm{Straight Line Tax Depreciation}=\frac{\textrm{Initial Cost}-\textrm{Scrap Value}}{t}}$$

		\subsubsection{Equivalent Annual Cost}

	$$\mathrm{EAC=\frac{NPV}{PVAF_{r,n}}}$$

		\subsubsection{Rate of Depreciation}

	$$\mathrm{r=r_B(1-T_C)}$$

	\subsection{Leases}

		\subsubsection{Net Advantage to Leasing}

	$$\mathrm{NAL=PV(\textrm{Cost to Lease})-PV(\textrm{Cost to Buy})}$$

\newpage

\section{Raising Equity}

	\subsection{Taking Up Rights}

		\subsubsection{Step 1}

	$$\mathrm{P_s=P_0(1-d)}$$

	$\mathrm{d=\textrm{Discount (Not Rate)}}$\\
	$\mathrm{P_s=\textrm{New Offer Share Price}}$\\
	$\mathrm{P_0=\textrm{Current Share Price}}$\\
	$\mathrm{P_x=\textrm{Share Price Day After Offer}}$

		\subsubsection{Step 2}

	$$\mathrm{N^*=\frac{F}{P_s}}$$

	$\mathrm{F=\textrm{Funds to Be Raised}}$\\
	$\mathrm{N^*=\textrm{Number of New Shares Issued}}$\\
	$\mathrm{N=\textrm{Number of Current Shares}}$
	
		\subsubsection{Step 3}

	$$\mathrm{\frac{N^*}{N}=\textrm{Ratio Offered}}$$

	$\mathrm{\textrm{To Lowest Denominator}}$\\
	``Offered N$^*$ (New) for Every N (Old)''

		\subsubsection{Step 4}

	$$\mathrm{P_{x(Pre-Issue)}=\frac{(P_0)(N)+F}{(N+N^*)}}$$

	$$\mathrm{P_x=\frac{(P_0)(N)+(P_s)(N^*)}{(N+N^*)}}$$

		\subsubsection{Step 5}

	$$\mathrm{\textrm{Rights Value}=P_x-P_s}$$

	$\mathrm{\textrm{If }P_x>P_s\textrm{: Capital Gain}}$\\
	$\mathrm{\textrm{If }P_x<P_s\textrm{: Capital Loss}}$

	\subsection{Selling Rights}

		\subsubsection{Step 1}

	\begin{itemize}
		\item Find Original Shares Owned: $$\mathrm{P_0N=x}$$
		\item Find Price to Sell New: $$\mathrm{P_x-P_s=\textrm{Rights Value}}$$
		\item Find Proportion Entitled To: $$\mathrm{\frac{N^*}{N}}$$
		\item Find Value of New: $$\mathrm{(P_x-P_s)N}$$
		\item Determine Cost: $$\mathrm{\textrm{Cost}=(P_0N)-((P_x-P_s)N^*)}$$ $$\mathit{\textrm{Should = }(P_0N)+(P_sN^*)}$$
	\end{itemize}

		\subsubsection{Step 2}
	
	\begin{itemize}
		\item Find Day-After Value of Only Current Shares: $$\mathrm{P_xN}$$
		\item Hence, Answers Should Be (=) Such That: \textit{``Value after selling new rights (=) value to buy current amount of shares owned, the day after''}
	\end{itemize}
	
	\subsection{Sell \& Take-Up (Tail Swallowing)}

	$$\mathrm{Y=\frac{(P_sN^*)}{P_x}}$$

	$\mathrm{Y=\textrm{Optimal Amount of Rights to Sell}}$\\
	$\mathrm{\textrm{Sell Newly Entitled Rights Proportion }Y\textrm{ to }P_s}$\\
	$\mathrm{\textrm{To Get Money for }(N-Y)\textrm{ New Shares At }P_x}$

		\subsubsection{Step 1 (Cost)}

	\begin{itemize}                                        
                \item Own $$\mathrm{N\textrm{ @ }P_0}$$
		\item Sell $$\mathrm{Y\textrm{ @ }(P_x-P_s)}$$
		\item Purchase $$\mathrm{(N^*-Y)\textrm{ @ }P_s}$$
	\end{itemize}

	$$\mathrm{\textrm{Cost}=P_0N+((P_x-P_s)Y)-(P_s(N^*-Y))}$$

	$\mathrm{P_0N=\textrm{Original}}$\\
	$\mathrm{(P_x-P_s)Y=\textrm{Sold Rights}}$\\
	$\mathrm{P_s(N^*-Y)=\textrm{Taken Rights}}$\\
	$\mathrm{(P_x-P_sY)\textrm{ Should = }P_s(N^*-Y)}$

		\subsubsection{Step 2 (Value)}

	$$\mathrm{\textrm{Own }(N+N^*-Y)\textrm{ @ }P_x }$$

	$$\mathrm{\therefore\textrm{Value}=P_x(N+N^*-Y)}$$

\end{document}

\documentclass[11pt, english]{article}              
        \usepackage{geometry}
                \geometry{                          
                        a4paper,total={210mm,297mm},
                        tmargin=40.8mm,
                        bmargin=40.8mm,
                        lmargin=32.6mm,
                        rmargin=32.6mm,
                }

        \usepackage{titlesec}         
                \titleformat{\section}
                        {\normalfont\fontsize{18}{16}\bfseries}{\thesection}{0.5em}{}
                \titleformat{\subsection}
                        {\normalfont\fontsize{14}{16}\bfseries}{\thesubsection}{1em}{}
                \titleformat{\subsubsection}
                        {\normalfont\fontsize{11}{16}\bfseries}{\thesubsubsection}{1em}{}

        \usepackage{longtable}
        \usepackage{multirow}

        \usepackage[labelfont=bf,textfont=bf,font=small,skip=8pt]{caption}

        \setlength{\parindent}{0pt}
        \renewcommand{\baselinestretch}{1.25}
       	\usepackage{setspace}

        \usepackage{amsmath}
        \usepackage{amssymb}

	\usepackage[utf8]{inputenc}
	\usepackage[official]{eurosym}

        \usepackage{graphicx}

\begin{document}

\pagenumbering{gobble}

        \title{\textsc{AG313 Treasury Management \& Derivatives\\ Coursework Summary}}
        \author{\textsc{Lewis Britton}}
        \date{\textsc{Academic Year 2019/2020}}
        \maketitle

\newpage

\pagenumbering{roman}

        \renewcommand{\contentsname}{Table of Contents}

        \tableofcontents

\newpage

\pagenumbering{arabic} 

\section{Derivatives}

	\subsection{Options}

		\subsubsection{Option vs. Forward Contracts}

	\begin{itemize}
	\setlength\itemsep{0cm}
		\item \textit{Option}: Right to buy/sell, in future, at rate (no future exchange rate saftey)
		\item \textit{Future}: Obligation to buy/sell, in future, at rate (future exchange rate safety)
	\end{itemize}

		\subsubsection{Spot vs. Future/Forward Prices}

	\begin{itemize}
        \setlength\itemsep{0cm}
		\item \textit{Spot Price}: immediate delivery ($\mathrm{S_0,S_T}$)
		\item \textit{Future/Forward Price}: future delivery price (locked-in today) ($\mathrm{F_0,F_T}$)
		\begin{itemize}
			\item $\mathrm{F_T<S_T}$: Forward = Spot grossed up @ r
			\item Spot expected to be $>$ r growth
		\end{itemize}
	\end{itemize}

		\subsubsection{Short vs. Long Positions}

	\begin{itemize}                                       
        \setlength\itemsep{0cm}
		\item \textit{Short}: Sell shares now ($\mathrm{S_0=Spot}$), buy later ($\mathrm{S_T=Delivery}$)
		\begin{itemize}
			\item Expect fall in share price, in future
			\item Futures price ($\uparrow$), loss
			\item Profit = $\mathrm{S_0-K}$
		\end{itemize}
		\item \textit{Long}: Buy shares now
		\begin{itemize}
			\item Expect rise in share price, in future
			\item Futures price ($\uparrow$), gain
			\item Profit = $\mathrm{K-S_T}$
		\end{itemize}
	\end{itemize}

		\subsubsection{Call vs. Put Options}

	\begin{itemize}
	\setlength\itemsep{0cm}
		\item ``At The Money'': $\mathrm{S_T=K}$
		\item \textit{Call}: Agreement to buy at specified time and Strike Price
		\begin{itemize}
			\item Profit (``In The Money''): $\mathrm{S_T>K}$
			\item Profit = $\mathrm{N(S_T-K)-Cost}$; $\mathrm{Cost=N(C_0)}$
		\end{itemize}
		\item \textit{Put}: Agreement to sell at specified time and Strike Price
		\begin{itemize} 
			\item Profit (``In The Money''): $\mathrm{K>S_T}$
			\item Profit = $\mathrm{Cost-N(K-S_T)}$; $\mathrm{Cost=N(P_0)}$
		\end{itemize}
		\item European Option: exercised only on expiration
		\item American Option: exercised any time up-to expiration and expiration
	\end{itemize}

		\subsubsection{Exchange vs. Over-the-Counter}

	\begin{itemize}                              
        \setlength\itemsep{0cm}
		\item \textit{Exchange}: \$60tn valuation; more standardized and regulated
		\begin{itemize}
			\item Trades Futures contracts
		\end{itemize}
		\item \textit{Over-the-Counter}: \$600tn valuation; higher credit risk, higher prices
		\begin{itemize}
			\item Trades Forward contracts
		\end{itemize}
	\end{itemize}

\newpage

	\subsection{Futures Markets}

	\begin{itemize}        
        \setlength\itemsep{0cm}
		\item Regulated by Commodities Futures Trading Commission (CFTC)
		\item \textit{Clearing House}: always used in Futures Market to ensure payment method
		\item \textit{Central Clearing Parties}: similar job to the above
		\item \textit{Haircut}: difference between Market Value and Collateral usage of an asset
		\item \textit{Bilateral Clearing}: group agree terms to trade w/ eash-other to minimise risk
		\item \textit{Limit Order}: trader identifies worst at which trade can take place
	\end{itemize}

		\subsubsection{Forward vs. Future}

	\begin{itemize}             
        \setlength\itemsep{0cm}
		\item Futures based on a shorter period than Forwards
		\item Futures usually don't have final cash settlements
	\end{itemize}

		\subsubsection{Margin `Curtain Call' Call}
	
	\begin{itemize}                           
        \setlength\itemsep{0cm}
		\item Broker's demand that investor adds funds to retain minimum value of fund, daily
		\item Options up-to 9 months must be bought in full; post-9-months margin can be taken
		\item The seller posts the margin
		\item Margin acounts adjusted daily for gain/loss
		\item Reduce systematic risk $\rightarrow$ ensure funds available $\rightarrow$ reduce risk of back-out
	\end{itemize}

	\begin{itemize}                                                
        \setlength\itemsep{0cm}
		\item Margin Call when: Loss $>$ (Initial Margin $-$ Maintenance Margin)
		\begin{itemize}
			\item[$\alpha$] If Short: ea. \$1 rise in price is a \$1 per-unit loss; find $=$ to above
			\item[$\beta$] Add the per-unit rise to the per-unit price
			\item[$\gamma$] If Long: ea. \$1 rise in price is a \$1 per-unit gain; find $=$ to above
			\item[$\delta$] Add the per-unit rise to the per-unit price
		\end{itemize}
	\end{itemize}

		\subsubsection{Corn Futures Contract}

	\begin{itemize}        
        \setlength\itemsep{0cm}
		\item Initiated by party w/ Short Position; `Notice of Intention' [to deliver]
		\item Exchange goes through procedure of choosing party to take Long Position
	\end{itemize}

		\subsubsection{Hedging vs. Speculating}

	\begin{itemize}        
        \setlength\itemsep{0cm}
		\item \textit{Hedging}: e.g. expect volatility, perhaps price rise to take Futures contract to lock in a better price now
		\item \textit{Speculating}: e.g. act upon volatility expectation perhaps where there's expected fall in price, take a Short position and buy back for profit
		\item Hedgers hold Long, Speculators hold Short: ($\mathrm{F_T>S_T}$)
	\end{itemize}

\newpage

	\subsection{Forward \& Futures Prices}

	\begin{itemize}        
        \setlength\itemsep{0cm}
		\item Future Price quoted as no. of US\$ per-unit of foreign currency
		\item Lenders cannot issue instructions
		\item \textit{Investment Asset}: traded but not usually phyisically usable or tangible
		\item \textit{Consumption Asset}: traded and usable for consumption (e.g. Copper)
		\item \textit{Convenience Yield}: 0/($+$), measures benefit of owning rather than Forward/Future
		\begin{itemize}
			\item Having real value vs. locked-in F value
			\item Investment: 0
			\item Consumption: ($+$)
			\item Increase: F as \% of S $\downarrow$; more convenieint to own
			\item Decrease: F as \% of S $\uparrow$; more convenient to F
		\end{itemize}
		\item \textit{Dividend Yield}: Div.'s as a \% of stock price at t of Div. payment
		\item \textit{Contango}: $\mathrm{F_T>S_T}$ abnormally
	\end{itemize}

		\subsubsection{Shorting With Dividends}

	\begin{enumerate}        
        \setlength\itemsep{0cm}
		\item Sell now ($\mathrm{S_0}$), buy later ($\mathrm{S_T}$) (Gain-Per-Share = $\mathrm{S_0-S_T}$)
		\item Pay Dividend (Gain-Per-Share = $\mathrm{S_0-S_T-Div.}$)
	\end{enumerate}

		\subsubsection{Spot-to-Forward Price}

	$$\mathrm{F_T=S_0e^{rT}}$$

	$$\mathrm{F_T=(S_0-Income)e^{rT}}$$
	$$\mathrm{Income=Y_te^{-rT}+...+Y_{t+n}e^{-rT}}$$

	$$\mathrm{F_T=ER_0e^{(r_1-r_2)T}}$$

\newpage

	\subsection{Hedging Strategies With Futures}

	\begin{itemize}                                                              
        \setlength\itemsep{0cm}
		\item Futures delivery month should be as close as possible to purchase of asset
		\item ``Tailing the Hedge'': corrects for daily settlement
		\item Hedging Futures leads to predictability
	\end{itemize}

	$$\mathrm{Basis=Spot_{At\ Close}-Futures_{At\ Close\ (For\ Maturity)}}$$

	$$\mathrm{Price\ Recieved=Basis+Futures_{At\ Purchase\ (For Maturity)}}$$

	$$\mathrm{Optimal\ Hedge\ Ratio=\rho_{A,B}\left(\frac{\sigma_A}{\sigma_B}\right)}$$
	\begin{center}``Movement in S price to movement in F price''\end{center}

	$$\mathrm{Optimal\ Folios=\left(\beta_{Current}-\beta_{Desired}\right)\left(\frac{V_{Folio}}{F_0F_N}\right)}$$
	\begin{center}If ($+$): Short\\ If ($-$): Long\end{center}

	$$\mathrm{P_{Total}=w_{Hedged}P_{Hedged}+w_{Not-Hedged}P_{Not-Hedged}}$$
	\begin{center}Where:\\ Given $\mathrm{S_0,F_0,S_T,F_T}$\\ $\mathrm{P_{Hedged}=S_T-(F_0F_T)}$\\ $\mathrm{P_{Not-Hedged}=S_T}$\end{center}
		
\newpage

	\subsection{Option Market Mechanics}	

	\begin{itemize}
        \setlength\itemsep{0cm}
		\item \textit{Option Class}: All Calls or Puts on a stock
		\item \textit{Option Series}: All options on a certain stock type
		\item \textit{LEAPS}: Long-Term Equity Anticipation Securities w/ long maturities
	\end{itemize}

	\begin{itemize}
        \setlength\itemsep{0cm}
		\item \textit{Stock Split}
		\begin{itemize}
			\item E.g.: N = 100, K = 20, 2-for-1 Split;
			\item Ans.: N = 2(100) = 200, K = $\frac{1}{2}$(20) = 10
		\end{itemize}
		\item \textit{Stock Dividend}
		\begin{itemize} 
			\item E.g.: N = 100, K = 20, Div. = 25\%;
			\item Ans.: N = 1.25(100) = 125, K = $\frac{4}{5}$(20) = 16
		\end{itemize}
		\item \textit{Cash Dividend}
		\begin{itemize}
			\item No effect
		\end{itemize}
	\end{itemize}

	\begin{itemize}                                                                 
        \setlength\itemsep{0cm}
		\item Option Value = Time Value $+$ Intrinsic Value
		\begin{itemize}
			\item At-the-Money Time Value = 0 so Option Value = Intrinsic Value
			\item Call: ($\mathrm{S_T-K,0}$)
			\item Put: ($\mathrm{K-S_T,0}$)
		\end{itemize}
	\end{itemize}

\newpage

	\subsection{Option Pricing}

		\subsubsection{Binomial Option Tree: European Put}

	\textbf{Step 1}

	$$\mathrm{u=e^{\sigma\sqrt{\Delta t}}}$$
	$$\mathrm{d=e^{-\sigma\sqrt{\Delta t}}=\frac{1}{u}}$$

	$$\mathrm{p=\frac{e^{r\Delta t}-d}{u-d}=Risk\ Neutral\ Probability\ of\ Up\ Movement}$$
	$$\mathrm{(1-p)=Risk\ Neautral\ Probability\ of\ Down\ Movement}$$

	\textbf{Step 2}

	$$\mathrm{S_{u/d}=Value\ of\ Stock\ Upon\ Increase/Decrease}$$
	$$\mathrm{S_u=Pu}$$
	$$\mathrm{S_d=Pd}$$
	$$\mathrm{S_{u,u}=Pu^2}$$
	$$\mathrm{S_{u,d}=Pud}$$
	$$\mathrm{S_{d,d}=Pd^2}$$

	\textbf{Step 3}

	$$\mathrm{P_{u/d}=Value\ of\ Option\ Upon\ Increase/Decrease}$$
	$$\mathrm{P_{u,u}=0}$$
	$$\mathrm{P_{u,d}=K-S_{u,d}}$$
	$$\mathrm{P_{d,d}=K-S_{d,d}}$$
	$$\mathrm{P_u=\left((pP_{u,u})+\left((1-p)P_{u,d}\right)\right)e^{-r\Delta t}}$$
	$$\mathrm{P_d=\left((pP_{u,d})+\left((1-p)P_{d,d}\right)\right)e^{-r\Delta t}}$$
	$$\mathrm{P_0=\left((pP_u)+\left((1-p)P_d\right)\right)e^{-r\Delta t}}$$

		\subsubsection{Binomial Option Tree: Converting to American Put}

	$$\mathrm{P_d=\max\{K-S_d,P_d\}}$$
	\begin{center}$\mathrm{P_{d_A}=Larger\ Outcome}$; $\mathrm{P_{u_A}=Remains\ Same}$\end{center}
	$$\mathrm{P_{0_A}=\left((pP_{u_A})+\left((1-p)P_{d_A}\right)\right)e^{-r\Delta t}}$$

\newpage

	\subsection{Stock Options}

	\begin{itemize}                                                                 
        \setlength\itemsep{0cm}
		\item Stock Price ($\uparrow$): Call ($\uparrow$); Put ($\downarrow$)
		\item Strike Price ($\uparrow$): Call ($\downarrow$); Put ($\uparrow$)
		\item Volatility ($\uparrow$): Call Payoff ($\uparrow$); Put Payoff ($\uparrow$)
		\item Dividends ($\uparrow$): Stock Price ($\downarrow$); Call ($\downarrow$); Put ($\uparrow$)
		\item Interest Rate ($\uparrow$): Call ($\uparrow$); Put ($\downarrow$)
		\item Time-Maturity ($\uparrow$): European Options ($\uparrow/\downarrow$)
	\end{itemize}

		\subsubsection{Call Lower-Bound}

	$$\mathrm{S_0-Ke^{-rT}}$$

		\subsubsection{Put Lower-Bound}

	$$\mathrm{Ke^{-rT}-S_0}$$

		\subsubsection{Put-Call Parity w/o Dividend (or 0 Interest)}

	$$\mathrm{C_0+Ke^{-rT}=P_0+S_0}$$
	$$\mathrm{C_0+K=P_0+S_0}$$

		\subsubsection{Put-Call Parity w/ Divided}

	$$\mathrm{C_0+Ke^{-rT}=P_0+(S_0-Div.)}$$

		\subsubsection{Black \& Scholes Models}

	$$\mathrm{d_1=\frac{\ln\left(\frac{S}{K}\right)+T\left(r+\frac{\sigma^2}{2}\right)}{\sigma\sqrt{T}}}$$
	$$\mathrm{d_2=d_1-\sigma\sqrt{T}}$$

	$$\mathrm{C_0=S(N(d_1))-Ke^{-rT}(N(d_2))}$$
	$$\mathrm{C_0=Se^{-yT}(N(d_1))-Ke^{-rT}(N(d_2))}$$
	$$\mathrm{P_0=K(1-N(d_1))-Se^{-rT}(1-N(d_2))}$$

\newpage

\section{Treasury Management}

	\subsection{Foreign Exchange Market}

	\textit{Domestic in terms of foreign; foreign in terms of domestic}

	$$\mathrm{Spread=\frac{Ask-Bid}{Ask}}$$

	$$\mathrm{Direct\ Quotation=\pounds/\$=\frac{1}{\$/\pounds}}$$

	$$\mathrm{Indirect\ Quotation=\$/\pounds=\frac{1}{\pounds/\$}}$$

	$$\mathrm{Cross\ Rate=\$/\pounds=EUR/\pounds\frac{1}{EUR/\$}}$$

	\subsection{Interest Parity Relationships}

		\subsubsection{Interest Rate Arbitrage}

	$$\mathrm{A_n=\left(\frac{A_h}{S}\right)(1+i_f)(S(1+p))}$$
	$$\mathrm{S(1+p)=F}$$

	$$\mathrm{A_{h,n}=Home/New\ Home\ Currency}$$
	$$\mathrm{i_{h,f}=Home/Foreign\ Currency}$$
	$$\mathrm{S=Spot\ Exchange\ Rate=N\ of\ \pounds\ Per\ Unit\ of\ \$}$$
	$$\mathrm{F=Forward\ (Locked)\ Exchange\ Rate=N\ of\ \pounds\ Per\ Unit\ of\ \$}$$
	$$\mathrm{p=Forward\ Premium=Amount\ By\ Which\ F\ is \uparrow/\downarrow Than\ S}$$

	$$\mathrm{Convert\ To\ \$:\ \left(\frac{A_h}{S}\right)}$$
	$$\mathrm{End\ of\ Period\ \$\ Principal\ \&\ Interest:\ \left(\frac{A_h}{S}\right)(1+i_f)}$$
	$$\mathrm{\$\ Principal\ \&\ Interest\ Back\ to\ \pounds:\ \left(\frac{A_h}{S}\right)(1+i_f)F}$$

		\subsubsection{Interest Rate No-Arbitrage}

	$$\mathrm{A_h(1+i_h)=A_h(1+i_f)(1+p)}$$
	$$\mathrm{A_h(1+i_h)=Investing\ w/\ Home\ Interest=Investing\ w/\ Foreign\ Interest\ w/\ p}$$
	$$\mathrm{\therefore p=\frac{(1+i_h)}{(1+i_f)}-1\therefore p\approx i_h-i_f}$$

		\subsubsection{Absolute PPP}

	$$\mathrm{S_f^d=\frac{P_s^d}{P_s^f}}$$
	$$\mathrm{As:\ P_s^d=S_f^dP_s^f}$$

		\subsubsection{Relative PPP w/ Inflation}

	$$\mathrm{P_h(1+\pi_h)}$$
	$$\mathrm{P_f(1+\pi_f)}$$

	\begin{center}
	If $\mathrm{\pi_h>\pi_f}$: PP is greater when buying foreign goods $\rightarrow$ foreign cheaper\\
	If $\mathrm{\pi_h<\pi_f}$: PP is greater when buying domestic goods $\rightarrow$
domestic cheaper
	\end{center}

	\begin{center}Adjust for Change in Currency:\end{center}
	$$\mathrm{P_f(1+\pi_f)(1+e_f)}$$
	$$\mathrm{e_f=\%\ Change\ Per\ Unit\ of\ Foreign\ Currency\ In\ Domestic\ Currency}$$

	\begin{center}Hence:\end{center}
	$$\mathrm{P_h(1+\pi_h)=P_f(1+\pi_f)(1+e_f)}$$
	$$\mathrm{e_f=\frac{P_h(1+\pi_h)}{P_f(1+\pi_f)}-1=\frac{(1+\pi_h)}{(1+\pi_f)}}$$

	\begin{center}Given $\mathrm{P_h=P_f}$:\\
	If $\mathrm{\pi_h>\pi_f}$: $\mathrm{e_f}$ ($+$): foreign should appreciate; domestic depreciate\\
	If $\mathrm{\pi_h<\pi_f}$: $\mathrm{e_f}$ ($-$): foreign should depreciate; domestic appreciate
	\end{center}

	\begin{center}For Relatively Low Inflation:\end{center}
	$$\mathrm{e_f=\frac{(1+\pi_h)}{(1+\pi_f)}-1\approx(\pi_h-\pi_f)}$$

\newpage

	\subsection{Exchange Exposure}

		\subsubsection{Variance of Two-Asset Folio}

	$$\mathrm{\sigma_{x,y}^2=\sigma_x^2+\sigma_y^2+2(cov_{x,y})}$$

	\begin{center}Hence:\end{center}
	$$\mathrm{p=\{x,y\}}$$
	$$\mathrm{cov_{x,y}=\rho_{x,y}\sigma_x\sigma_y}$$

	$$\mathrm{\sigma_p^2=\sigma_x^2+\sigma_y^2+2(\rho_{x,y}\sigma_x\sigma_y)}$$

		\subsubsection{Variance of Three-Asset Folio}

	$$\mathrm{\sigma_p^2=\sigma_x^2+\sigma_y^2+\sigma_z^2+2(\rho_{x,y}\sigma_x\sigma_y)+2(\rho_{x,z}\sigma_x\sigma_z)+2(\rho_{y,z}\sigma_y\sigma_z)}$$

		\subsubsection{Economic Exposure}

	$$\mathrm{V_{MNC}=\sum\frac{\sum(E(CF_{j,t})E(ER_{j,t}))}{(1+k)^t}}$$

	\begin{center}Where:\end{center}
	$$\mathrm{E(CF_{j,t})=Expected\ CF\ in\ Currency\ j\ Recieved\ At\ End\ of\ Period\ t}$$
	$$\mathrm{E(ER)_{j,t}=Expeced\ ER\ of\ Currency\ j\ At\ End\ of\ Peiod\ t}$$
	$$\mathrm{k=Weighted\ Average\ Cost\ of\ Capital\ (WACC)\ of\ MNC}$$

\newpage

	\subsection{Value of A Multinational Corporation}

	\begin{itemize}
	\setlength\itemsep{0cm}
		\item Value of Parent Company ($\mathrm{p}$, perhaps in USD)
		\item Value of Subsidiary 1 ($\mathrm{s1}$, perhasp in EUR)
		\item Value of Subsidiary 2 ($\mathrm{s2}$, perhaps in GBP)
	\end{itemize}

		\subsubsection{Basic Values}
	
	$$\mathrm{V_t=\frac{E(C_{t+1})}{(1+r)^{t+1}}+\frac{E(C_{t+2})}{(1+r)^{t+2}}+\frac{E(C_{t+3})}{(1+r)^{t+3}}}$$

	Value of Cash Flows in USD \textit{Functional Currency}:
	$$\mathrm{V_{t,p}=\frac{E(C_{t+1,\$})}{(1+r_\$)^{t+1}}+\frac{E(C_{t+2,\$})}{(1+r_\$)^{t+2}}+...+\frac{E(C_{t+n,{\$}})}{(1+r_\$)^{t+n}}}$$

	Value of Cash Flows in EUR:
	$$\mathrm{V_{t,s1}=\frac{E(C_{t+1,EUR})}{(1+r_EUR)^{t+1}}+\frac{E(C_{t+2,EUR})}{(1+r_EUR)^{t+2}}+...+\frac{E(C_{t+n,{EUR}})}{(1+r_EUR)^{t+n}}}$$

	Value of Cash Flows in GBP:
	$$\mathrm{V_{t,s2}=\frac{E(C_{t+1,\pounds})}{(1+r_\pounds)^{t+1}}+\frac{E(C_{t+2,\pounds})}{(1+r_\pounds)^{t+2}}+...+\frac{E(C_{t+n,{\pounds}})}{(1+r_\pounds)^{t+n}}}$$

		\subsubsection{Value Exchnage Conversion}

	Value of Cash Flows in USD \textit{Functional Currency}:
	$$\mathrm{V_{t,p}=\frac{E\left(C_{t+1,\$}\left(\frac{\$}{\$}\right)_{t+1}\right)}{(1+r_\$)^{t+1}}+\frac{E\left(C_{t+2,\$}\left(\frac{\$}{\$}\right)_{t+2}\right)}{(1+r_\$)^{t+2}}+...+\frac{E\left(C_{t+n,{\$}}\left(\frac{\$}{\$}\right)_{t+3}\right)}{(1+r_\$)^{t+n}}}$$

	Value of Cash Flows in USD \textit{Converted from EUR}:
	$$\mathrm{V_{t,p}=\frac{E\left(C_{t+1,EUR}\left(\frac{\$}{EUR}\right)_{t+1}\right)}{(1+r_\$)^{t+1}}+\frac{E\left(C_{t+2,EUR}\left(\frac{\$}{EUR}\right)_{t+2}\right)}{(1+r_\$)^{t+2}}+...}$$ $$\mathrm{+\frac{E\left(C_{t+n,{EUR}}\left(\frac{\$}{EUR}\right)_{t+3}\right)}{(1+r_\$)^{t+n}}}$$

	Value of Cash Flows in USD \textit{Converted from GBP}:
	$$\mathrm{V_{t,p}=\frac{E\left(C_{t+1,\pounds}\left(\frac{\$}{\pounds}\right)_{t+1}\right)}{(1+r_\$)^{t+1}}+\frac{E\left(C_{t+2,\pounds}\left(\frac{\$}{\pounds}\right)_{t+2}\right)}{(1+r_\$)^{t+2}}+...+\frac{E\left(C_{t+n,{\pounds}}\left(\frac{\$}{\pounds}\right)_{t+3}\right)}{(1+r_\$)^{t+n}}}$$

		\subsubsection{Value of Each Domestic/Foreign Operation}

	Hence \textit{Total Value of Parent Corporation (p)}:
	$$\mathrm{V_{t,p}=\sum_{i=1}^n\frac{E(C_{t+i,\$})}{(1+r_\$)^{t+i}}}$$

	Hence \textit{Total Value of European Subsidiary (s1)}:
	$$\mathrm{V_{t,s1}=\sum_{i=1}^n\frac{E\left(C_{t+i,EUR}\left(\frac{\$}{EUR}\right)_{t+i}\right)}{(1+r_\$)^{t+i}}}$$

	Hence \textit{Total Value of British Subsidairy (s2)}:
	$$\mathrm{V_{t,s2}=\sum_{i=1}^n\frac{E\left(C_{t+i,\pounds}\left(\frac{\$}{\pounds}\right)_{t+i}\right)}{(1+r_\$)^{t+i}}}$$

		\subsubsection{Total Value of Multinational Corporation}

	Hence \textit{Total Value of Multinational Corporation}:
	$$\mathrm{V_p=V_{s1}+V_{s2}}$$

	Hence \textit{For 3 Currency Example Over n Periods (i)}:
	$$\mathrm{V_{t,MNC}=\sum_{i=1}^n\frac{E(C_{t+i,\$})}{(1+r_\$)^{t+i}}+\sum_{i=1}^n\frac{E\left(C_{t+i,EUR}\left(\frac{\$}{EUR}\right)_{t+i}\right)}{(1+r_\$)^{t+i}}+\sum_{i=1}^n\frac{E\left(C_{t+i,\pounds}\left(\frac{\$}{\pounds}\right)_{t+i}\right)}{(1+r_\$)^{t+i}}}$$

	Hence \textit{Generalised for 3 Unknown Currencies Over n Periods (i)}:
	$$\mathrm{V_{t,MNC}=\sum_{i=1}^n\frac{E\left(C_{t+i,1}(ER_1)_{t+i}\right)}{(1+r_\$)^{t+i}}+\sum_{i=1}^n\frac{E\left(C_{t+i,2}(ER_2)_{t+i}\right)}{(1+r_\$)^{t+i}}+\sum_{i=1}^n\frac{E\left(C_{t+i,3}(ER_3)_{t+i}\right)}{(1+r_\$)^{t+i}}}$$

	Hence \textit{Generalised for n Unkown Currencies (j) Over n Periods (i)}: 
	$$\mathrm{V_{t,MNC}=\sum_{j=1}^n\left(\sum_{i=1}^n\frac{E\left(C_{t+i,j}(ER_j)_{t+i}\right)}{(1+r_\$)^{t+i}}\right)}$$

		\subsubsection{How Can The Value Change?}

	If $\mathrm{C_{t+i,j}<E(C_{t+i})_j}$: $\mathrm{V_{t,MNC}}$ lower than expected \textit{country business risk}\\
If $\mathrm{r_\$>E{r_\$}}$: $\mathrm{V_{t,MNC}}$ lower than expected \textit{country policy risk}\\
If $\mathrm{(ER_\$)_{t+i}<(ER_{\pounds})_{t+i}}$: $\mathrm{V_{t,MNC}}$ lower than expected \textit{foreign exchange risk (where \$ is domestic)}

\newpage

	\subsection{Interest Rate Risk}

	\begin{itemize}
	\setlength\itemsep{0pt}
		\item $\frac{1}{100}$ of a \%pt. is a `Basis Point'
		\item Must convert period to days
	\end{itemize}

	$$\mathrm{R=Simple\ Interest\ Rate}$$
	$$\mathrm{r=\frac{R}{m}=Periodic\ Interest\ Rate}$$
	\begin{center}``m periods per n''\end{center}
	$$\mathrm{\left(1+r\right)^{mn}-1=Compound\ Interest\ Rate}$$
	$$\mathrm{EAR=(1+r)^\frac{year}{days}-1}$$

		\subsubsection{Duration}

	$$\mathrm{\Delta B=-DB\Delta y}$$
	$$\mathrm{B=\sum\frac{{CF}_t}{\left(1+y\right)^t}}$$
	$$\mathrm{D=\sum t\left(\frac{\frac{CF_t}{(1+y)^t}}{P}\right)=\sum tw_t}$$

	$$\mathrm{y=Yield\ on\ Bond}$$
	$$\mathrm{P=Bond\ Price}$$

	$$\mathrm{D_{Zero-Coupon}=Maturity=T}$$
	$$\mathrm{Constant\ Maturity:\ D(\uparrow),\ CF(\downarrow)}$$
	$$\mathrm{Constant\ Coupon:\ D(\uparrow),\ T(\uparrow)}$$
	$$\mathrm{Constant\ All\ Other:\ D(\uparrow),\ y(\downarrow)}$$

		\subsubsection{Forward Rate Agreements}

	$$\mathrm{Payoff=\left(Notional\ Amount\right)\left(LIBOR-Agreed\ Upon\ Rate\right)\left(\frac{m}{360}\right)}$$

	$$\mathrm{Payoff=\left(Notional\ Amount\right)\left((\left(LIBOR\right)-Agreed\ Upon\ Rate)\frac{\left(\frac{m}{360}\right)}{\left(1+LIBOR\right)\left(\frac{m}{360}\right)}\right)}$$

		\subsubsection{Interest Rate Option}

	$$\mathrm{{\rm Payoff}_{Call}=\left(Notional\ Amount\right)\left(Max\left(0,LIBOR-X\right)\left(\frac{m}{360}\right)\right)}$$
	$$\mathrm{If\ LIBOR>X:\ Exercise}$$
	$$\mathrm{Payoff(\uparrow),\ LIBOR(\uparrow)}$$
	$$\mathrm{Protection\ Against\ Rising\ i\ (e.g.\ future\ borrowing)}$$

	$$\mathrm{{\rm Payoff}_{Put}=\left(Notional\ Amount\right)\left(Max\left(0,X-LIBOR\right)\left(\frac{m}{360}\right)\right)}$$                               
        $$\mathrm{If\ LIBOR<X:\ Exercise}$$
        $$\mathrm{Payoff(\uparrow),\ LIBOR(\downarrow)}$$
        $$\mathrm{Protection\ Against\ Falling\ i\ (e.g.\ future\ investing)}$$

\end{document}
